\documentclass[9pt,a4paper]{article}


\usepackage[utf8]{inputenc}
\usepackage{amsthm}
\usepackage{mathtools}
\usepackage{amsfonts}

\title{Resumen matemática discreta 1}
\author{Andrés Villagra de la Fuente - {\bfseries  andres.villagra(at)alumnos.unc.edu.ar}}

\theoremstyle{definition}
\newtheorem{defi}{Definicion}
\newtheorem{axi}{Axioma}

\theoremstyle{plain}
\newtheorem{teo}{Teorema}
\newtheorem{demo}{Demostración}[teo]
\newtheorem{col}{Colonario}
\newtheorem{obs}{Observación}
\begin{document}

\maketitle

\part{Números Enteros}

\section{Introducción}

\begin{defi} Los números naturales son un conjunto con una {\bfseries función  sucesor} que satisface los siguientes axiomas: \end{defi}

\begin{axi} \label{natual1}1 es un número Natural. \end{axi}
\begin{axi} \label{suc}Todo número tiene un sucesor\end{axi}
\begin{axi} \label{sucind}La función sucesor es injectiva ($\forall x \forall y, suc(x) \neq suc(y)$ si $x \neq y$)\end{axi}
\begin{axi} \label{primer1}El 1 no es sucesor de nadie\end{axi}
\begin{axi} \label{cnmp}si $S$ es un conjunt con una función sucesor, $1 \in S$ y satisface los axiomas \ref{natural1}-\ref{primer1} entonces es un conjunto natural\end{axi}

Por el axioma \ref{cnmp} podemos ver que el conjunto de los naturales es el conjunto de números mas pequeño con estás propiedades.

\begin{obs}
\begin{equation*}
\underbrace{|\mathbb{N}\cup\{\dots;\frac{-3}{2};\frac{-2}{2};\frac{-1}{2};\frac{1}{2};\frac{2}{2};\frac{3}{2};\dots\}}_\text{Satisface A\ref{natural1}-A\ref{primer1} pero no A\ref{cnmp}}|>|\mathbb{N}|
\end{equation*}
\end{obs}

\section{Aritmética}

En esta sección daremos definición a las diferentes operaciónes aritméticas y sus diferentes propiedades.

Para definir la suma damos uso del principio de inducción (definición \ref{defindux}) para definirla en $\mathbb{N}$

\begin{defi} \label{defsuma} 
\begin{align*}
\text{suma}(m,n) &\doteq \text{ si } n=1, \text{suc}(m) \\
                 &\doteq \text{ si no } n = \text{suc}(n-1) \implies \text{suma}(m,n-1)                
\end{align*}
\end{defi}

\begin{defi}\label{defmq} si $m,n\in\mathbb{N}$, decimos que $n<m$ si $\exists J\in\mathbb{N}$ tal que $m=n+J$\end{defi}

\begin{teo} si $m,n\in\mathbb{N}$, entonces $n>m \lor n<m \implies m\neq n$\end{teo}

\begin{demo}

\begin{align*}
\text{Si }& m<n \land m=n; m,n \in{\mathbb{N}} \\
          & \exists L \text{ tal que } n=m+L   \\
          & \text{Ya que } m=n \implies L=0    \\
	  & L\notin\mathbb{N}                  \\           
	  & \text{Absurdo.}
\end{align*}

\end{demo}

Para definir el producto damos uso tambien de una definición recursiva

\begin{defi} \label{defprod}
\begin{align*}
\text{prod}(m,n) &\doteq \text{si } n=1    \implies \text{prod}(m,1)=m \\
                 &\doteq \text{si } n\neq1 \implies \text{prod}(m,n)=\text{prod}(m,n-1)+m
\end{align*}
\end{defi}

\begin{obs}
Al definir así podesmo ver tambien que:\\
prod$(m,n)=$ prod$(m,n-1)+m$\\
Por ende la propiedad distributiva
\end{obs}

\part{Conteo}


\part{Divisibilidad}

\part{Aritmética Modular}

\part{Números Complejos}

\part{Grafos}

\end{document}
