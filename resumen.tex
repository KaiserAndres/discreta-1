\documentclass[9pt,a4paper]{article}


\usepackage[utf8]{inputenc}
\usepackage{amsthm}
\usepackage{mathtools}
\usepackage{amsfonts}

\title{Resumen matemática discreta 1}
\author{Andrés Villagra de la Fuente - {\bfseries  andres.villagra(at)alumnos.unc.edu.ar}}

\theoremstyle{definition}
\newtheorem{defi}{Definición}
\newtheorem{axi}{Axioma}
\newtheorem{lema}{Lema}

\theoremstyle{plain}
\newtheorem{teo}{Teorema}
\newtheorem{demo}{Demostración}[teo]
\newtheorem{col}{Colonario}
\newtheorem{obs}{Observación}
\begin{document}

\maketitle

\part{Números Enteros}

\section{Introducción}

\begin{defi} Los números naturales son un conjunto con una {\bfseries función  sucesor} que satisface los siguientes axiomas: \end{defi}

\begin{axi} \label{natual1}1 es un número Natural. \end{axi}
\begin{axi} \label{suc}Todo número tiene un sucesor\end{axi}
\begin{axi} \label{sucind}La función sucesor es injectiva ($\forall x \forall y, suc(x) \neq suc(y)$ si $x \neq y$)\end{axi}
\begin{axi} \label{primer1}El 1 no es sucesor de nadie\end{axi}
\begin{axi} \label{cnmp}si $S$ es un conjunt con una función sucesor, $1 \in S$ y satisface los axiomas \ref{natural1}-\ref{primer1} entonces es un conjunto natural\end{axi}

Por el axioma \ref{cnmp} podemos ver que el conjunto de los naturales es el conjunto de números mas pequeño con estás propiedades.

\begin{obs}
\begin{equation*}
\underbrace{|\mathbb{N}\cup\{\dots;\frac{-3}{2};\frac{-2}{2};\frac{-1}{2};\frac{1}{2};\frac{2}{2};\frac{3}{2};\dots\}}_\text{Satisface A\ref{natural1}-A\ref{primer1} pero no A\ref{cnmp}}|>|\mathbb{N}|
\end{equation*}
\end{obs}

\section{Aritmética}

En esta sección daremos definición a las diferentes operaciónes aritméticas y sus diferentes propiedades.

Para definir la suma damos uso del principio de inducción (definición \ref{defindux}) para definirla en $\mathbb{N}$

\begin{defi} \label{defsuma} 
\begin{align*}
\text{suma}(m,n) &\doteq \text{ si } n=1, \text{suc}(m) \\
                 &\doteq \text{ si no } n = \text{suc}(n-1) \implies \text{suma}(\text{suc}(m),n-1)                
\end{align*}
\end{defi}

\begin{defi}\label{defmq} si $m,n\in\mathbb{N}$, decimos que $n<m$ si $\exists J\in\mathbb{N}$ tal que $m=n+J$\end{defi}

\begin{teo} si $m,n\in\mathbb{N}$, entonces $n>m \lor n<m \implies m\neq n$\end{teo}

\begin{demo}

\begin{align*}
\text{Si }& m<n \land m=n; m,n \in{\mathbb{N}} \\
          & \exists L \text{ tal que } n=m+L   \\
          & \text{Ya que } m=n \implies L=0    \\
	  & L\notin\mathbb{N}                  \\           
	  & \text{Absurdo.}
\end{align*}

\end{demo}

Para definir el producto damos uso tambien de una definición recursiva.

\begin{defi} \label{defprod}
\begin{align*}
\text{prod}(m,n) &\doteq \text{si } n=1    \implies \text{prod}(m,1)=m \\
                 &\doteq \text{si } n\neq1 \implies \text{prod}(m,n)=\text{prod}(m,n-1)+m
\end{align*}
\end{defi}

\begin{obs} \label{obsdist}
Al definir así podesmo ver tambien que:\\
prod$(m,n)=$ prod$(m,n-1)+m$\\
Por ende la propiedad distributiva
\begin{align*}
prod(a,suma(b,c))=&suma(prod(a,b),prod(a,c))\\
&a+prod(a,suma(b,c)-1)\\
&\text{por siguiente}\\
&prod(a,b)+prod(a,sum(b,c)-b)\\
&prod(a,b)+prod(a,c)
\end{align*}
\end{obs}

\begin{lema}
Si $m,n,l\in\mathbb{N}$, tal que $l<m\land m<n\implies l<n$ (transitiva)

\begin{align*}
& m = a+l   & a\in\mathbb{N} \\
& n = b+m   & b\in\mathbb{N} \\
& n = \underbrace{b+a}_{\in{\mathbb{N}}}+l \implies n<l
\end{align*}
\end{lema}

\section{inducción}

\begin{axi}\label{defindN} Principio de inducción\\
Suponiendo que $\forall x\in{\mathbb{N}}$ tenemos una afirmación $P(x)$, y queremos provar su veracidad.\\
Si valen:\\
$1^{er}$ caso: $P(1)$ es verdadera
Paso Inductivo: si $P(x)$ es verdadera $\implies P(x+1)$ tambien debe ser verdadera $\forall x\in{\mathbb{N}}$
\end{axi}

El principio de inducción nos permite demostrar que una proposición es valida en todo el conjunto de $\mathbb{N}$, además podemos comprobar una proposición en $\mathbb{Z}$ (ver teo. \ref{defindZ})

\begin{defi} Sucesiones recursivas:\\
Dado $(a_n), n\in\mathbb{N}$ una sucesión de números.\\
$$S_k = \sum_{i=1}^{k}a_i = a_1 + a_2 + \dots + a_k \forall k\in{\mathbb{N}}$$
\end{defi}

Un ejemplo de definición de sucesión recursiva

\begin{align*}
S_1 = & a_1 \\
S_k = & S_{k-1} + a_k & k>1
\end{align*}

Afirmo que está bien definida:\\

$$S=\{k\in{\mathbb{N}}: S_k \text{ esta bien definida}\}$$

Quiero probar que $S=\mathbb{N}$

\begin{align*}
& 1 \in S \\
& \text{Si } k\in{S}\implies k+1\in{S}\\
& \underbrace{\mathbb{N}\subseteq S \subseteq\mathbb{N}}_{\text{Por el A\ref{cnmp}}\implies S=\mathbb{N}}
\end{align*}

\begin{defi} Inducción corrida \\ {\bfseries TODO}  \end{defi}

\begin{teo} \label{teobuenorden}Todo subconjunto no vacio de $\mathbb{N}$ tiene primer (o mínimo) elemento \end{teo}

\begin{demo}Bucar una buena demo\end{demo}

\section{Propiedades de $\mathbb{N}$}

$(\mathbb{N}, +, \bullet)$ es un {\itshape trinoide} y satisface las siguientes propiedades:

$(\mathbb{N}, +)$ vale:

\begin{teo} \label{teoasosum}Es asociativo:\\
\begin{align*} & (a+b)+c=a+(b+c) & \forall a,b,c\in{\mathbb{N}} \end{align*}
\end{teo}

\begin{demo} Tomamos como hipótesis inductiva que:

\begin{align*} & (l+m)+n=l+(m+n) & \forall l,m\in{\mathbb{N}}\end{align*}

$P(1):$
\begin{align*}
(l+m)+1 & = l+(m+1)
        & = \text{suma}(l,m+1) \\
        & = \text{suc}(\text{suma}(l,+m)) \\
        & = \text{suma}(l,m)+1 \\
        & = (l+m)+1 
\end{align*}

Ahora demostramos para $n+1$

$P(n+1):$
\begin{align*}
l+(m+(n+1)) & = l+\text{suma}(m, n+1) \\
            & = l+\text{suc}(\text{suma}/(m,n)) \\
            & = \text{suc}(l+\text{suma}(m,n)) \\
            & = \text{suc}(l+m+n) \\
            & = \text{suc}((l+m)+n) \\
            & = (l+m)+(n+1)
\end{align*}
\end{demo}

\begin{teo} \label{teoconmusum} Es conmutativa:\\
\begin{align*} & m+n=n+m & \forall m,n\in{\mathbb{N}} \end{align*}
\end{teo}

\begin{demo}

Al igual que la anterior daremos uso de una demostración por inducción, con la siguiente hipótesis inductiva:\\

$$m+n=n+m$$

$P(1)$

\begin{align*}
m+1       &= 1+m\\
suma(m,1) &= suma(1,m)\\
suc(m)    &= suma(1,m)\\
suc(m)    &= suma(suc(1), m-1)\\
\text{siguiendo por $m-2$ pasos}\\
          & \vdots\\
suc(m)    &= suma(m, 1)\\
suc(m)    &= suc(m)
\end{align*}

$P(n+1)$

\begin{align*}
m+(n+1)         &= (n+1)+m\\
suma(m,n+1)     &= suma(n+1,m)\\
suma(m, suc(n)) &= suma(suc(n), m)\\
suc(suma(m,n))  &= suma(m,suc(n)) \text{ Por el T\ref{teoasosum}}\\
suc(suma(m,n))  &= suc(suma(m,n))
\end{align*}
\end{demo}

$(\mathbb{N}, \bullet)$ vale:

\begin{teo} \label{teoasopro}Es asociativa:\\
\begin{align*} & l*(m*n)=(l*m)*n & \forall l,m,n\in{\mathbb{N}} \end{align*}
\end{teo}

\begin{demo}
Para demostrar esto daremos uso de la Obs.\ref{obsdist} y de la siguiente hipótesis inductiva:\\

$$l*(m*n)=(l*m)*n$$

$P(1)$
\begin{align*}
l*(m*1) &= (l*m)*1\\
prod(l,prod(m,1)) &= prod(prod(l,m), 1)\\
prod(l,m) &= prod(l,m)
\end{align*}

$P(n+1)$

\begin{align*}
l*(m*(n+1)) = (l*m)*(n+1)\\
A := l*(m*(n+1))\\
B := (l*m)*(n+1)\\
\end{align*}
A)
\begin{align*}
l*&(m*(n+1))\\
prod&(l,prod(m,n+1))\\
prod&(l,prod(m,suma(n,1)))\\
\text{Por uso de la Obs.\ref{obsdist}}\\
prod&(l, prod(m,n)+prod(m,1))\\
prod&(l, suma(prod(m,n),m))\\
\text{Por uso de la Obs.\ref{obsdist}}\\
A := suma&(prod(l, prod(m,n)),prod(l,m))
\end{align*}
B)
\begin{align*}
(l*m)*&(n+1)\\
prod&(prod(l,m),suma(n,1))\\
\text{Por uso de la Obs.\ref{obsdist}}\\
suma&(prod(prod(l,m),n),prod(prod(l,m),1))\\
B := suma&(prod(prod(l,m),n),prod(l,m))\\
\end{align*}
\begin{align*}
A &= B\\
suma(prod(l, prod(m,n)),prod(l,m)) &= suma(prod(l, prod(m,n)),prod(l,m))
\end{align*}
\end{demo}

\begin{teo} \label{teomonpro}Es conmutativo:\\
\begin{align*} & l*m=m*l & \forall l,m\in{\mathbb{N}} \end{align*}
\end{teo}

\begin{teo} \label{teoelneu}Existe el elemento neutro:\\
\begin{align*} & 1*l=l & \forall l\in{\mathbb{N}} \end{align*}
\end{teo}

Todos estos teoremas se pueden demostrar usando {\itshape inducción en $\mathbb{N}$} 

TODO: demostrar estos teoremas.

\section{Inducción fuerte}

La inducción fuerte o completa es una forma de inducción que, como el nombre indica, da uso de una hipótesis mas fuerte. Mientras que la inducción debil toma como hipótesis que si $P(1)$ y $P(n+1)$ son verdaderas entonces $P$ es verdadero $\forall{n}\in{\mathbb{N}}$, la inducción fuerte toma que:\\

\begin{defi} Si $P(n)$ es verdadero $\forall{n} {1}\leq{n}\leq{k}$ y también $P(n+1)$ es verdadero $\forall n>k$, entonces $P$ es verdadero.\end{defi}

\begin{teo} Principio del buen orden\\

Si $A \subseteq \mathbb{N}, A \neq \emptyset$, entonces $A$ tiene primer elemento. (minimo)
\end{teo}

\begin{demo} Suponemos que $k\in{A}, {1}\leq{k}\leq{n}$, entonces $A$ tiene primer elemento, y veamos que si $(n+1)\in{A}$, $A$ tiene primer elemento.\\

Casos:\\
1) Si $1\notin{A}, \land \dots \land n\notin{A} \implies A$ tiene primer elemento al saber $n+1$\\
2) Si hay un $k\in{\mathbb{N}}, 1\leq{k}\leq{n}, k\in{A} \implies P(k)\text{ si } k\in{A}, A$ tiene primer elemento por hipótesis inductiva.
\end{demo}

\section{El conjunto de $\mathbb{Z}$}

\begin{defi} El conjunto de los enteros está definido como:\\
$$\mathbb{Z}:= \mathbb{N}\cup{\{0\}\cup{-\mathbb{N}}}$$\end{defi}

Este conjunto opera de manera similar al de los naturales, especialemte en operaciónes donde ambas partes pertenecen a $\mathbb{N}$, sin embargo al trabajar con elementos que no esten en $\mathbb{N}$ estas operaciónes deben redefinirse, esto nos ofrece eliminar ciertas limitaciones expandiendo el conjunto de operaciónes posibles para el conjunto.

\begin{defi} $$\forall{m,n,l}\in{\mathbb{N}}, m>n\implies m-n=l / m=n+l$$

\begin{align*}
\text{resta}(m,n) &:= \text{ si } n=1: \text{prev}(m)\\
                  &:= \text{ si } n\neq{1}: \text{resta}(\text{prev}(m), \text{prev(n)})
\end{align*}
\end{defi}

\begin{obs}La resta {\bfseries no} es una operación en $\mathbb{N}$, sino en $\mathbb{Z}$\end{obs}

\begin{defi}$(\mathbb{Z}, +)$\\
$\bullet \text{ si }a,b\in{\mathbb{N}}$, las anteriores.\\
$\bullet \text{ si }a\in{\mathbb{Z}}, a+0=a$ \\
$\bullet$ si \[a\in{-\mathbb{N}}, b\in{-\mathbb{N}} a+b=
\begin{cases}
\text{ si } & a>-b\\
            &a-(-b)\\
\text{ si } &a=-b\\
            &0\\
\text{ si } &a<-b\\
            &-(-b-a)
\end{cases}
\]
$\bullet \text{ si }a,b\in{-\mathbb{N}} a+b=((-a)+(-b))$
\end{defi}

\begin{defi} $(\mathbb{Z}, *)$\\
$\bullet\text{ si } a,b\in{\mathbb{N}}\text{ las anteriores}$\\
$\bullet\text{ si } a\in{\mathbb{Z}} a*0=0$\\
$\bullet\text{ si } a\in{\mathbb{N}}, b\in{-\mathbb{N}} a*b=-(a*(-b))$\\
$\bullet\text{ si } a,b\in{-\mathbb{N}} a*b=(-a)*(-b)$
\end{defi}

El conjunto $(\mathbb{Z}, +, *)$ satisface las siguientes propiedades:\\

1) $(\mathbb{Z}, +)$ es:
\begin{enumerate}
\item Asociativa \label{zsumaso}
\item Conmutativa \label{zsumacon}
\item Elemento neutro: 0 \label{zsumneu}
\item Inverso: si $a\in{\mathbb{Z}}$, existe $b\in{\mathbb{Z}}/a+b=0 \land b=-a$ \label{zsuminv}
\end{enumerate}

Todo conjunto que satisface \ref{zsumaso} y \ref{zsuminv} se llama {\itshape abeliano}. \\

2) $(\mathbb{Z}, *)$ satisface:\\
\begin{enumerate}
\item Asociativa
\item Conmutativa
\item Elemento neutro 1: $a*1=a\forall{a}\in{\mathbb{Z}}$
\end{enumerate}

3) Distributiva $a*(b+c)=a*b+a*c$\\

Un grupo que satisface \ref{zsumaso}, \ref{zsumacon} y \ref{zsumneu} se llama {\itshape conmutativo}. 

\begin{obs} Es posible usar como notación para los números en $\mathbb{Z}$ como una tupla de números en $\mathbb{N}$.
\begin{align*}
\mathbb{Z} &= \mathbb{N}\times\mathbb{N}\\
a,b &\in{\mathbb{N}}\\
a-b &:= (a,b)=(c,d)\iff a+d=b+c
\end{align*}
Como ejemplo:
\begin{align*}
0 &= (1,1),(2,2) \text{ etc}\dots\\
1 &= (2,1),(3,2) \text{ etc}\dots\\
\end{align*}
Esta forma de notar los numeros en $\mathbb{Z}$ permite también operar:
\begin{align*}
(a,b)+(c,d)&=(a+c,b+d)\\
(a,b)+(c,d)&=(a*c+b*d,a*d+b*c)
\end{align*}
\end{obs}

\part{Conteo}

\part{Divisibilidad}

\part{Aritmética Modular}

\part{Números Complejos}

\part{Grafos}

\end{document}
