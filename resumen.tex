\documentclass[9pt,a4paper,draft]{article}

\usepackage[utf8]{inputenc}
\usepackage{amsthm}
\usepackage{amsmath}
\usepackage{mathtools}
\usepackage{amsfonts}
\usepackage{amssymb}

\title{Resumen matemática discreta 1}
\author{Andrés Villagra de la Fuente - {\bfseries  andres.villagra(at)alumnos.unc.edu.ar}}

\theoremstyle{definition}
\newtheorem{defi}{Definición}
\newtheorem{axi}{Axioma}
\newtheorem{lema}{Lema}

\theoremstyle{plain}
\newtheorem{teo}{Teorema}
\newtheorem{demo}{Demostración}[teo]
\newtheorem{col}{Colonario}
\newtheorem{obs}{Observación}

\DeclareMathOperator{\Div}{Div}
\DeclareMathOperator{\mcd}{mcd}
\DeclareMathOperator{\mcm}{mcm}
\DeclareMathOperator{\sig}{signo}
\DeclareMathOperator{\re}{Re}
\DeclareMathOperator{\im}{Im}

\newcommand{\modu}[1]{\quad(\text{mód }#1)}
\begin{document}

\maketitle

\part{Números Enteros}

\section{Introducción}

\begin{defi} Los números naturales son un conjunto con una {\bfseries función  sucesor} que satisface los siguientes axiomas: \end{defi}

\begin{axi} \label{natual1}1 es un número Natural. \end{axi}
\begin{axi} \label{suc}Todo número tiene un sucesor\end{axi}
\begin{axi} \label{sucind}La función sucesor es injectiva ($\forall x \forall y, suc(x) \neq suc(y)$ si $x \neq y$)\end{axi}
\begin{axi} \label{primer1}El 1 no es sucesor de nadie\end{axi}
\begin{axi} \label{cnmp}si $S$ es un conjunt con una función sucesor, $1 \in S$ y satisface los axiomas \ref{natural1}-\ref{primer1} entonces es un conjunto natural\end{axi}

Por el axioma \ref{cnmp} podemos ver que el conjunto de los naturales es el conjunto de números mas pequeño con estás propiedades.

\begin{obs}
\begin{equation*}
\underbrace{|\mathbb{N}\cup\{\dots;\frac{-3}{2};\frac{-2}{2};\frac{-1}{2};\frac{1}{2};\frac{2}{2};\frac{3}{2};\dots\}}_\text{Satisface A\ref{natural1}-A\ref{primer1} pero no A\ref{cnmp}}|>|\mathbb{N}|
\end{equation*}
\end{obs}

\section{Aritmética}

En esta sección daremos definición a las diferentes operaciónes aritméticas y sus diferentes propiedades.

Para definir la suma damos uso del principio de inducción (definición \ref{defindux}) para definirla en $\mathbb{N}$

\begin{defi} \label{defsuma} 
\begin{align*}
\text{suma}(m,n) &\doteq \text{ si } n=1, \text{suc}(m) \\
                 &\doteq \text{ si no } n = \text{suc}(n-1) \implies \text{suma}(\text{suc}(m),n-1)                
\end{align*}
\end{defi}

\begin{defi}\label{defmq} si $m,n\in\mathbb{N}$, decimos que $n<m$ si $\exists J\in\mathbb{N}$ tal que $m=n+J$\end{defi}

\begin{teo} si $m,n\in\mathbb{N}$, entonces $n>m \lor n<m \implies m\neq n$\end{teo}

\begin{demo}

\begin{align*}
\text{Si }& m<n \land m=n; m,n \in{\mathbb{N}} \\
          & \exists L \text{ tal que } n=m+L   \\
          & \text{Ya que } m=n \implies L=0    \\
	  & L\notin\mathbb{N}                  \\           
	  & \text{Absurdo.}
\end{align*}

\end{demo}

Para definir el producto damos uso tambien de una definición recursiva.

\begin{defi} \label{defprod}
\begin{align*}
\text{prod}(m,n) &\doteq \text{si } n=1    \implies \text{prod}(m,1)=m \\
                 &\doteq \text{si } n\neq1 \implies \text{prod}(m,n)=\text{prod}(m,n-1)+m
\end{align*}
\end{defi}

\begin{obs} \label{obsdist}
Al definir así podesmo ver tambien que:\\
prod$(m,n)=$ prod$(m,n-1)+m$\\
Por ende la propiedad distributiva
\begin{align*}
prod(a,suma(b,c))=&suma(prod(a,b),prod(a,c))\\
&a+prod(a,suma(b,c)-1)\\
&\text{por siguiente}\\
&prod(a,b)+prod(a,sum(b,c)-b)\\
&prod(a,b)+prod(a,c)
\end{align*}
\end{obs}

\begin{lema}
Si $m,n,l\in\mathbb{N}$, tal que $l<m\land m<n\implies l<n$ (transitiva)

\begin{align*}
& m = a+l   & a\in\mathbb{N} \\
& n = b+m   & b\in\mathbb{N} \\
& n = \underbrace{b+a}_{\in{\mathbb{N}}}+l \implies n<l
\end{align*}
\end{lema}

\section{inducción}

\begin{axi}\label{defindN} Principio de inducción\\
Suponiendo que $\forall x\in{\mathbb{N}}$ tenemos una afirmación $P(x)$, y queremos provar su veracidad.\\
Si valen:\\
$1^{er}$ caso: $P(1)$ es verdadera
Paso Inductivo: si $P(x)$ es verdadera $\implies P(x+1)$ tambien debe ser verdadera $\forall x\in{\mathbb{N}}$
\end{axi}

El principio de inducción nos permite demostrar que una proposición es valida en todo el conjunto de $\mathbb{N}$, además podemos comprobar una proposición en $\mathbb{Z}$ (ver teo. \ref{defindZ})

\begin{defi} Sucesiones recursivas:\\
Dado $(a_n), n\in\mathbb{N}$ una sucesión de números.\\
$$S_k = \sum_{i=1}^{k}a_i = a_1 + a_2 + \dots + a_k \forall k\in{\mathbb{N}}$$
\end{defi}

Un ejemplo de definición de sucesión recursiva

\begin{align*}
S_1 = & a_1 \\
S_k = & S_{k-1} + a_k & k>1
\end{align*}

Afirmo que está bien definida:\\

$$S=\{k\in{\mathbb{N}}: S_k \text{ esta bien definida}\}$$

Quiero probar que $S=\mathbb{N}$

\begin{align*}
& 1 \in S \\
& \text{Si } k\in{S}\implies k+1\in{S}\\
& \underbrace{\mathbb{N}\subseteq S \subseteq\mathbb{N}}_{\text{Por el A\ref{cnmp}}\implies S=\mathbb{N}}
\end{align*}

\begin{defi} Inducción corrida \\ {\bfseries TODO}  \end{defi}

\begin{teo} \label{teobuenorden}Todo subconjunto no vacio de $\mathbb{N}$ tiene primer (o mínimo) elemento \end{teo}

\begin{demo}Bucar una buena demo\end{demo}

\section{Propiedades de $\mathbb{N}$}

$(\mathbb{N}, +, \bullet)$ es un {\itshape trinoide} y satisface las siguientes propiedades:

$(\mathbb{N}, +)$ vale:

\begin{teo} \label{teoasosum}Es asociativo:\\
\begin{align*} & (a+b)+c=a+(b+c) & \forall a,b,c\in{\mathbb{N}} \end{align*}
\end{teo}

\begin{demo} Tomamos como hipótesis inductiva que:

\begin{align*} & (l+m)+n=l+(m+n) & \forall l,m\in{\mathbb{N}}\end{align*}

$P(1):$
\begin{align*}
(l+m)+1 & = l+(m+1)
        & = \text{suma}(l,m+1) \\
        & = \text{suc}(\text{suma}(l,+m)) \\
        & = \text{suma}(l,m)+1 \\
        & = (l+m)+1 
\end{align*}

Ahora demostramos para $n+1$

$P(n+1):$
\begin{align*}
l+(m+(n+1)) & = l+\text{suma}(m, n+1) \\
            & = l+\text{suc}(\text{suma}/(m,n)) \\
            & = \text{suc}(l+\text{suma}(m,n)) \\
            & = \text{suc}(l+m+n) \\
            & = \text{suc}((l+m)+n) \\
            & = (l+m)+(n+1)
\end{align*}
\end{demo}

\begin{teo} \label{teoconmusum} Es conmutativa:\\
\begin{align*} & m+n=n+m & \forall m,n\in{\mathbb{N}} \end{align*}
\end{teo}

\begin{demo}

Al igual que la anterior daremos uso de una demostración por inducción, con la siguiente hipótesis inductiva:\\

$$m+n=n+m$$

$P(1)$

\begin{align*}
m+1       &= 1+m\\
suma(m,1) &= suma(1,m)\\
suc(m)    &= suma(1,m)\\
suc(m)    &= suma(suc(1), m-1)\\
\text{siguiendo por $m-2$ pasos}\\
          & \vdots\\
suc(m)    &= suma(m, 1)\\
suc(m)    &= suc(m)
\end{align*}

$P(n+1)$

\begin{align*}
m+(n+1)         &= (n+1)+m\\
suma(m,n+1)     &= suma(n+1,m)\\
suma(m, suc(n)) &= suma(suc(n), m)\\
suc(suma(m,n))  &= suma(m,suc(n)) \text{ Por el T\ref{teoasosum}}\\
suc(suma(m,n))  &= suc(suma(m,n))
\end{align*}
\end{demo}

$(\mathbb{N}, \bullet)$ vale:

\begin{teo} \label{teoasopro}Es asociativa:\\
\begin{align*} & l*(m*n)=(l*m)*n & \forall l,m,n\in{\mathbb{N}} \end{align*}
\end{teo}

\begin{demo}
Para demostrar esto daremos uso de la Obs.\ref{obsdist} y de la siguiente hipótesis inductiva:\\

$$l*(m*n)=(l*m)*n$$

$P(1)$
\begin{align*}
l*(m*1) &= (l*m)*1\\
prod(l,prod(m,1)) &= prod(prod(l,m), 1)\\
prod(l,m) &= prod(l,m)
\end{align*}

$P(n+1)$

\begin{align*}
l*(m*(n+1)) = (l*m)*(n+1)\\
A := l*(m*(n+1))\\
B := (l*m)*(n+1)\\
\end{align*}
A)
\begin{align*}
l*&(m*(n+1))\\
prod&(l,prod(m,n+1))\\
prod&(l,prod(m,suma(n,1)))\\
\text{Por uso de la Obs.\ref{obsdist}}\\
prod&(l, prod(m,n)+prod(m,1))\\
prod&(l, suma(prod(m,n),m))\\
\text{Por uso de la Obs.\ref{obsdist}}\\
A := suma&(prod(l, prod(m,n)),prod(l,m))
\end{align*}
B)
\begin{align*}
(l*m)*&(n+1)\\
prod&(prod(l,m),suma(n,1))\\
\text{Por uso de la Obs.\ref{obsdist}}\\
suma&(prod(prod(l,m),n),prod(prod(l,m),1))\\
B := suma&(prod(prod(l,m),n),prod(l,m))\\
\end{align*}
\begin{align*}
A &= B\\
suma(prod(l, prod(m,n)),prod(l,m)) &= suma(prod(l, prod(m,n)),prod(l,m))
\end{align*}
\end{demo}

\begin{teo} \label{teomonpro}Es conmutativo:\\
\begin{align*} & l*m=m*l & \forall l,m\in{\mathbb{N}} \end{align*}
\end{teo}

\begin{teo} \label{teoelneu}Existe el elemento neutro:\\
\begin{align*} & 1*l=l & \forall l\in{\mathbb{N}} \end{align*}
\end{teo}

Todos estos teoremas se pueden demostrar usando {\itshape inducción en $\mathbb{N}$} 

TODO: demostrar estos teoremas.

\section{Inducción fuerte}

La inducción fuerte o completa es una forma de inducción que, como el nombre indica, da uso de una hipótesis mas fuerte. Mientras que la inducción debil toma como hipótesis que si $P(1)$ y $P(n+1)$ son verdaderas entonces $P$ es verdadero $\forall{n}\in{\mathbb{N}}$, la inducción fuerte toma que:\\

\begin{defi} Si $P(n)$ es verdadero $\forall{n} {1}\leq{n}\leq{k}$ y también $P(n+1)$ es verdadero $\forall n>k$, entonces $P$ es verdadero.\end{defi}

\begin{teo} Principio del buen orden\\

Si $A \subseteq \mathbb{N}, A \neq \emptyset$, entonces $A$ tiene primer elemento. (minimo)
\end{teo}

\begin{demo} Suponemos que $k\in{A}, {1}\leq{k}\leq{n}$, entonces $A$ tiene primer elemento, y veamos que si $(n+1)\in{A}$, $A$ tiene primer elemento.\\

Casos:\\
1) Si $1\notin{A}, \land \dots \land n\notin{A} \implies A$ tiene primer elemento al saber $n+1$\\
2) Si hay un $k\in{\mathbb{N}}, 1\leq{k}\leq{n}, k\in{A} \implies P(k)\text{ si } k\in{A}, A$ tiene primer elemento por hipótesis inductiva.
\end{demo}

\section{Inducción corrida}

La inducción corrida es un caso especial de la inducción estandard, en la cual toma como hipótesis:

\begin{defi} Dado un $k\in{\mathbb{N}}$ y una proposición $P$, se dice que $P(n)$ es verdadero si:\\
$$P(k) \text{ es verdadero.}$$
$$P(n+1) \forall{n>k} \text{ es verdadero.}$$
\end{defi}

\section{El conjunto de $\mathbb{Z}$}

\begin{defi} El conjunto de los enteros está definido como:\\
$$\mathbb{Z}:= \mathbb{N}\cup{\{0\}\cup{-\mathbb{N}}}$$\end{defi}

Este conjunto opera de manera similar al de los naturales, especialemte en operaciónes donde ambas partes pertenecen a $\mathbb{N}$, sin embargo al trabajar con elementos que no esten en $\mathbb{N}$ estas operaciónes deben redefinirse, esto nos ofrece eliminar ciertas limitaciones expandiendo el conjunto de operaciónes posibles para el conjunto.

\begin{defi} $$\forall{m,n,l}\in{\mathbb{N}}, m>n\implies m-n=l / m=n+l$$

\begin{align*}
\text{resta}(m,n) &:= \text{ si } n=1: \text{prev}(m)\\
                  &:= \text{ si } n\neq{1}: \text{resta}(\text{prev}(m), \text{prev(n)})
\end{align*}
\end{defi}

\begin{obs}La resta {\bfseries no} es una operación en $\mathbb{N}$, sino en $\mathbb{Z}$\end{obs}

\begin{defi}$(\mathbb{Z}, +)$\\
$\bullet \text{ si }a,b\in{\mathbb{N}}$, las anteriores.\\
$\bullet \text{ si }a\in{\mathbb{Z}}, a+0=a$ \\
$\bullet$ si \[a\in{-\mathbb{N}}, b\in{-\mathbb{N}} a+b=
\begin{cases}
\text{ si } & a>-b\\
            &a-(-b)\\
\text{ si } &a=-b\\
            &0\\
\text{ si } &a<-b\\
            &-(-b-a)
\end{cases}
\]
$\bullet \text{ si }a,b\in{-\mathbb{N}} a+b=((-a)+(-b))$
\end{defi}

\begin{defi} $(\mathbb{Z}, *)$\\
$\bullet\text{ si } a,b\in{\mathbb{N}}\text{ las anteriores}$\\
$\bullet\text{ si } a\in{\mathbb{Z}} a*0=0$\\
$\bullet\text{ si } a\in{\mathbb{N}}, b\in{-\mathbb{N}} a*b=-(a*(-b))$\\
$\bullet\text{ si } a,b\in{-\mathbb{N}} a*b=(-a)*(-b)$
\end{defi}

El conjunto $(\mathbb{Z}, +, *)$ satisface las siguientes propiedades:\\

1) $(\mathbb{Z}, +)$ es:
\begin{enumerate}
\item Asociativa \label{zsumaso}
\item Conmutativa \label{zsumacon}
\item Elemento neutro: 0 \label{zsumneu}
\item Inverso: si $a\in{\mathbb{Z}}$, existe $b\in{\mathbb{Z}}/a+b=0 \land b=-a$ \label{zsuminv}
\end{enumerate}

Todo conjunto que satisface \ref{zsumaso} y \ref{zsuminv} se llama {\itshape abeliano}. \\

2) $(\mathbb{Z}, *)$ satisface:\\
\begin{enumerate}
\item Asociativa
\item Conmutativa
\item Elemento neutro 1: $a*1=a\forall{a}\in{\mathbb{Z}}$
\end{enumerate}

3) Distributiva $a*(b+c)=a*b+a*c$\\

Un grupo que satisface \ref{zsumaso}, \ref{zsumacon} y \ref{zsumneu} se llama {\itshape conmutativo}. 

\begin{obs} Es posible usar como notación para los números en $\mathbb{Z}$ como una tupla de números en $\mathbb{N}$.
\begin{align*}
\mathbb{Z} &= \mathbb{N}\times\mathbb{N}\\
a,b &\in{\mathbb{N}}\\
a-b &:= (a,b)=(c,d)\iff a+d=b+c
\end{align*}
Como ejemplo:
\begin{align*}
0 &= (1,1),(2,2) \text{ etc}\dots\\
1 &= (2,1),(3,2) \text{ etc}\dots\\
\end{align*}
Esta forma de notar los numeros en $\mathbb{Z}$ permite también operar:
\begin{align*}
(a,b)+(c,d)&=(a+c,b+d)\\
(a,b)+(c,d)&=(a*c+b*d,a*d+b*c)
\end{align*}
\end{obs}

\part{Conteo}

\section{Cardinalidad de conjuntos}

\begin{obs}
Para el proposito de este texto usaremos $|A|$ para denotar la función {\itshape cardinalidad} de un conjunto, sin embargo tambien se puede usar $\#A$ o $tam(A)$.
\end{obs}

Para contar los elementos de un conjunto, les asignamos un número natural ordenado a cada uno de los elementos del conjunto, para esto damos uso de una función $\dot{f}$ biyectiva.

$$\{1,2,3,4\}\xrightarrow{\dot{f}}A$$

¿Porque esta operación es correcta? En general si $n\neq{m}\, n,m\in{\mathbb{N}}\, \text{ no existe }\{1\dots n\}\to\{1\dots m\}$\\

\begin{teo}
En el caso que tengamos dos conjuntos {\bfseries disjuntos} $A$ y $B$, se da que:\\
$$|A|+|B|=|A\cup B|$$
Y tambíen se da en el caso del producto cruz entre los conjuntos.\\
$$|A|*|B|=|A\times{B}|$$
\end{teo}

\begin{demo} Asumiendo que $A\cap{B}=\emptyset$
\begin{align*}
& |A| = n & A=\{a_1 \dots a_n\}\\
& |B| = m & B=\{b_1 \dots b_m\}\\
& A\cup{B}=\{\underset{1}a_1\dots \underset{n}a_n, \underset{n+1}b_1 \dots \underset{n+m}b_m\} & \implies |A\cup{B}|=m+n
\end{align*}
\end{demo}

\begin{teo} Si $A_1 \dots A_n$ son conjuntos finitos y disjuntos dos a dos ($A_i\cap A_j = \emptyset,\text{ si }i\neq{j}$) entonces $|A_1\cup{A_2}\cup{\dots}\cup{A_n}| = |A_1|+|A_2|+\dots+|A_n|$\end{teo}

\begin{demo}
Daremos uso de inducción con la hipótesis inductiva propuesta anteriormente.\\
$P(1)$ es verdadera ya que $|A| = |A|$\\
$P(n+1)$:\\
$$|\underbrace{A_1\cup{A_2}\cup{\dots}\cup{A_n}}_{B}\cup{A_{n+1}}|=|B\cup{A_{n+1}}|=|B|+|A_{n+1}|$$

Si $x\in{B\cap{A_{n+1}}}, x\in{B}\land x\in{A_{n+1}}\iff x\in{A_i}$ para algun $i$, $1\leq{i}\leq{n} \land x\in{A_{n+1}} \implies x\in A_i \cap A_{n+1} \neq \emptyset$.
\end{demo}

¿Que es lo que sucede si eliminamos a asumpción de que $A\cap{B}=\emptyset$?

\begin{align*}
A          &= \{A \backslash(A\cap{B})\}\cup{(A\cap{B})}\\
B          &= \{B \backslash(A\cap{B})\}\cup{(A\cap{B})}\\
A\cup{B}   &= \{A\backslash(A\cap{B})\}\cup{\{A\cap{B}\}}\cup \{B\backslash(A\cap{B})\}\\
|A\cup{B}| &= |A\backslash(A\cap{B})| + |A\cap{B}| + |B\backslash(A\cap{B})|\\
|A|        &= |A\cap{B}| + |A\backslash(A\cap{B})|\\
|A| - |A\cap{B}| &= |A\backslash(A\cap{B})|\\
|A\cup{B}| &= |A| - |A\cap{B}| + |A\cap{B}| + |B| - |A\cap{B}|\\
|A\cup{B}| &= |B| + |B| - |A\cap{B}|
\end{align*}

\begin{teo} Si $A$ y $B$ son conjuntos finitos, $|A\times{B}| = |A|*|B|$\end{teo}
TODO: desmostrar.

\section{Conteo y combinatoria.}

El conteo como el nombre indica consiste en contar las diferentes combinaciónes especificas de un conjunto según alguna condición dada. Por ejemplo para ver todos los números de 5 dijitos capicua podemos usar el proceso siguiente:

$$\underbrace{*+-}_\text{unica variación de importancia}+*$$

Podemos tomar unicamente los primeros tres digitos, esto se da por la simetría de los números capicua.

$$|{1\dots 9}\times{0\dots 9}\times{0\dots 9}| = 9*10*10 = 900$$

En general se puede pensar en este tipo de problemas como:

$$|\{f\{1\dots n\}\to\{1\dots m\}, \text{ injectivas}\}|$$

\section{Numero binomial o combinatorio}

\begin{defi} EL número combinatorio lo denotamos como: $$\binom{m}{n}=\frac{m!}{(m-n)!*n!}$$\end{defi}

Dado un conjunto de $m$ elementos, cuantos subjoncjuntos de $n$ elementos existen? $\binom{m}{n}$ ya que cada conjunto lo contamos $n!$ veces por las distintas permutaciónes, por lo que cada combinación aparece $n!$ veces.

\begin{defi} Complemento del número combinatorio: $$\binom{m}{n}=\binom{m}{n-1}$$\end{defi}

\begin{teo} si $m,n\in{\mathbb{N}}\;1\leq{n}\leq{m}$ $$\binom{m+1}{n}=\binom{m}{n} + \binom{m}{n-1}$$ \end{teo}

\begin{demo}
Quiero contar conjuntos $A$ con $n$ elementos

\begin{itemize}
\item Subconjuntos $A\backslash m+1\in{A}$
\item Subconjuntos $A\backslash m+1\notin{A}$
\end{itemize}

Si $m+1\in{A}\implies$ resta elejur $n+1$ elementos de $m$, lo puedo hacer de $\binom{m}{n-1}$.\\

Si $m+1\notin{A}\implies$ resta elejir $n$ elementos de $m$ cosas $\binom{m}{n}$

$$\text{Por ende hay } \binom{m}{n} + \binom{m}{n-1}$$

\end{demo}

\begin{teo} Binomio de newton: Si $a,b\in{\mathbb{R}},\, n\in{\mathbb{N}}$
$$(a+b)^n = \displaystyle\sum_{i=0}^{n} \binom{n}{i}*a^i*b^{n-i}$$
\end{teo}

TODO: demostrar?

\part{Divisibilidad}

En $\mathbb{Z}$ vimos que hay un producto con cirtas propiedades.

\begin{defi} si $a,b\in{\mathbb{Z}}$, decimos que $a$ "divide" a $b$ (denotado como $a|b$) si $\exists{q}\backslash b=a*q$\end{defi}

Para $a,b,c\in{\mathbb{Z}}$ valen las siguientes propiedades:

\begin{teo} si $a|b\implies a|b*c$ \end{teo}
\begin{demo}si $a|b\implies b=a*q$, por lo que $b*c=a*k$\end{demo}

\begin{teo} si $a|b$ y $b|c\implies a|c$\end{teo}
\begin{demo}$a|b\implies b=a*q,\; b|c\implies c=b*k\implies c=\underbrace{(a*q)}_{b}*k$\end{demo}

\begin{teo} \label{diva0} si $a|b$ y $b\neq{0}\implies|a|\leq|b|$\end{teo}
\begin{demo}
si $a|b\implies b=a*q,\;q\in{\mathbb{Z}}$
\begin{align*}
\text{Como } b\neq{0}&, q\neq{0}, |q|\geq{1}\\
|b|&=|a*q|=|a|*|q|\\
|q|&\geq{1}\\
|a|*|q|&\geq|a|
\end{align*}
\end{demo}

\begin{teo}\label{divDsuma} si $a|b$ y $a|c\implies{a|b+c}$, mas generalmente $m,n\in{\mathbb{Z}},\;a|m*b+n*c$ \end{teo}
\begin{demo}
si $a|b\implies b=a*q,\;a|c\implies c=a*p$

\begin{align*}
b+c&=a*q+a*p\\
 &=a*(q+p) \implies a|b+c
\end{align*}
Mas generalmente
\begin{align*}
m*b+n*c &= m*a*q+n*a*p\\
 &= a*(m*q+n*p) \implies a|m*b+n*c
\end{align*}
\end{demo}

TODO: añadir referencia entre la identidad de pascal y el triangulo de pascal.

\begin{defi} si $a\in{\mathbb{Z}}$ definimos el conjunto de divisores de $a$ como:
\begin{align*}
\Div(a) &= \{b\in{\mathbb{Z}} \backslash b|a\}\\\
\Div(2) &= \{-1,-1,1,2\}
\end{align*}
\end{defi}

\begin{obs} Si $a|b \implies -a|b \; (b=a*q,\, b=(-a)*(-q))$ \\ $$\Div(0) = \mathbb{Z} \; 0|0 \iff 0=0*1$$
Si $a\in{\mathbb{Z}}, a\neq{0}, \Div(a)$ es finito porque el teorema \ref{diva0} dice que si $a\neq{0}, b|a, |b|\leq|a|\implies \Div(a)\subset \{b\in{\mathbb{Z}} \backslash |b|\leq|a|\}$
\end{obs}

\begin{defi} Si $a,b\in{\mathbb{Z}}$, alguno no nulo, definimos el "maximo común divisor" de $a$ y $b$ como  el mas grande de los diisores comúnes, o sea:
$$\operatorname{max}(\{\Div(a)\cap\Div(b)\}) = \mcd(a,b)$$

Analogamente:

\begin{enumerate}
\item $\mcd(a,b), mcd(a,b)|b$
\item si $d|a$ y $d|b,\; d\leq{\mcd(a,b)}$
\end{enumerate}
\end{defi}

\begin{teo} si $a,b\in{\mathbb{Z}}$, algúno no nulo, $\mcd(a,b) = \mcd(a, b-a*m)\forall m\in{\mathbb{Z}}$\end{teo}

\begin{teo} $$\Div(a)\cap\Div(B)=\Div(a)\cap\Div(b-a*m)$$
$$A=B \iff A\subseteq{B} \land B\subseteq{A}$$
Lutego podemos ver que:
$$\Div(a)\cap\Div(b) \subseteq \Div(a)\cap\Div(b-a*m)$$
Si $d\in{\Div(a)\cap\Div(b)},\; d|a \land d|b \implies d|a\; d|b-a*m$ por Teo \ref{divDsuma}, y $\Div(a)\cap\Div(b-a*m)\subseteq\Div(a)\cap\Div(b)$

$$d|a \land d|b-a*m \implies d|a\;d|(b-a*m)+a\;m=b \text{ (Teo. \ref{divDsuma})}$$
\end{teo}

\begin{teo} dados $a,b\in{\mathbb{Z}}$ con $a\neq{0}$ existen {\bfseries únicos} $q,r\in{\mathbb{Z}},\; 0\leq{r}\leq{|a|}\text{ tal que } b=a*q+r$ \end{teo}

\begin{demo} {\bfseries Existencia}

Dado un conjunto $\mathcal{R} = \{ b-a*q\; q\in{\mathbb{Z}}\}\cap\{\mathbb{Z}\cup\{0\}\}$ \\
$r$ es el menor elemento de $\mathcal{R}$ \\
Veamos que $\mathcal{R}\neq\emptyset$

\begin{enumerate}
\item Si $b\geq0$, con $q=0, b-a*0 = b \geq 0$
\item si $b<0$,
\begin{align*}
b+|a*b| &\geq 0\\
a \neq 0\\
|a*b| &\geq |b|\\
b+|a*b| &\geq |b|+b = 0\\
\text{Tomando } q &= b * \sig(a) = |a|*|b|\\
-a*q = -a*b*\sig(a) &= |a|*|b|
\end{align*}
\end{enumerate}

Como $\mathcal{R}\neq{\emptyset},\; r=b-a*q \iff b=a*q+r$\\
Falta ver que $0\leq{r}\leq{|a|}$\\
Si $r\geq{|a|}, r-|a|\geq{0}$, pero $r-|a|<r$ (pues $|a|>0$) lo que contradice que r es el mismo.\\

{\bfseries Unicidad} 

\begin{align*}
b &= a*q+r & 0\leq{r}, & \tilde{r} < |a|\\
b &= a*\tilde{q}+\tilde{r}
\end{align*}

Quiro ver que $r=\tilde{r}, q=\tilde{q}$
$$0=a*(q-\tilde{q})+r-\tilde{r}\implies r-\tilde{r}=a*(\tilde{q}-q)$$
Si $r-\tilde{r}\neq0,\; |a|\leq|r-\tilde{r}|$

\begin{align*}
0\leq{r}<|a| & 0\leq{r}<|a| & 0\leq{r}<|a|\\
0\leq{\tilde{r}}<|a| &\implies 0\geq-\tilde{r}>|a| & -|a|<\tilde{r}\leq{0}\\
-|a|<r-\tilde{r}<|a| &\implies |r-\tilde{r}| < |a|
\end{align*}

Esto es una contradiccioón, con lo cual $r-\tilde{r}=0\implies r=\tilde{r}$
$$a*q=b-r=b-\tilde{r}=a*\tilde{q}\implies q=\tilde{q}$$
\end{demo}

\begin{teo} $a,b\in{\mathbb{Z}}, a\neq{0},\; a|b$ el resto de dividir $b$ por $a$ es 0\end{teo}
\begin{demo} 
Si $r=0 \implies b=a*q+0 \implies a|b$ \\
Reciprocamente si $a|b,\; b=a*q=a*q+\underbrace{0}_\text{Unico posible resto}$
\end{demo}

\section{Bases Númericas}

$\underbrace{1}_{1000}\underbrace{2}_{100}\underbrace{3}_{10}\underbrace{4}_{1}$, este es un sistema posicional con base 10, donde 1 decena = 10 unidades, 1 centena = 10 decenas = $10^2$ unidades, etc...\\

Así: $1237 = 1*10^3+2*10^2+3*10+7*10^0$

¿Como podemos escribir un número en base $b$?

\begin{align*} & n=\displaystyle\sum_{i=0}^{m}a_i*b^i & a_i=(0, \dots, b-1)\end{align*}

\begin{teo} Si $a\in{\mathbb{N}}m d\geq2, n\in{\mathbb{N}}, n$ se puede escribir de manera única como:
$$n = a_1 + a_1*d^1 + a_2*d^2 + \dots + a_m*d^m$$
Donde $0\leq{a_i}<d$
\end{teo}
\begin{demo}
Si $n=1, n=1$ ya está.\\
Supongamos que vale hasta $n$, veamos que $n+1$:\\
$n+1$ dividido $d, n+1=d*q+r$ de forma unica.\\
Por H.I. $q=b_0+b_1*d+\dots+b_j*d^j$ de forma única.\\
$$n+1 = r + b_0*d + b_1*d^2 + \dots + b_j*d^{j+1}$$
\end{demo}

\begin{teo} \label{teoeuclides} Si $a,b\in{\mathbb{Z}}$, alguno no núlo, entonces existen $m,n\in{\mathbb{Z}}$ que:\\
\begin{align*} & \mcd(a,b) = a*m+b*n & |a|\geq|b| \end{align*}
El máximo común divisor se puede escribir como la combinación lineal de $a$ y $b$.
\end{teo}

TODO: demostrar.

\section{Ecuaciónes diafanticas}

\begin{teo}\label{defdiafantica} si $a,b,c \in{\mathbb{Z}}$, la ecuación lienas:
$$ax+by=c$$
Tiene soluciones enteras $\iff \mcd(a,b)|c$. En tal caso las soluciones son infinatas, dadas por:
\[
\begin{cases}
x = x_0 + \frac{b}{\mcd(a,b)}t \\
y = y_0 + \frac{a}{\mcd(a,b)} t 
\end{cases}
\]
\end{teo}
\begin{demo} Tenemos que demostrar ambas partes de $\iff$\\
\begin{itemize}
\item $\Rightarrow) \mcd(a,b)$ divide al lado iziquiero de la igualdad $\forall{x,y}\in{\mathbb{Z}} \implies$ si has solución $\mcd(a,b)|c$ \\
\item $\Leftarrow) \exists{r,s}\in{\mathbb{Z}}$ tal que $\mcd(a,b) = \underbrace{ar+bj}_{\oplus}$ \\
$\mcd(a,b)|c \implies c = \mcd(a,b)*q,\; q\in{\mathbb{Z}}$\\
Multiplicamos $\oplus$ por $q$\\
$q*\mcd(a,b) = c = a\underbrace{ar}+b\underbrace{qj}\; (qr, qj)$ es solución.
\end{itemize}
\end{demo}

\begin{defi} si $n\in{\mathbb{Z}}, n$ es primo si: $\#\Div(n)=4$ \\
(Tiene exactamente 4 divisores distintos: $\pm1, \pm n$)\end{defi}

\begin{obs}
1 {\bfseries no} es primo. 
\end{obs}

\begin{obs}
si $n$ no es primer, $n\neq{1,0}$ se dice compuesto. Vale que $n=ab\; a,b\neq\pm1$\\
si $d|n, |d|\leq{n}$
\end{obs}

\begin{teo} \label{teofundamentalaritmetica} Todo $n\in{\mathbb{Z}}, n\neq{0,1,-1}$ se escribe como
$$n ) \sig(n)\prod_{i=1}^{r}p_i^{e_i}$$
Donde $p_i$ son primos positivos distintos, $e_i\in{\mathbb{N}}$
\end{teo}
\begin{demo}
Podemos suponer que $n>0 \; (n\in{\mathbb{N}})$, entonces por inducción fuerte:\\
$n=2=2^1$, supongamos que vale $i=1\dots n$, veamos $n+1$
\begin{itemize}
\item Si $n+1$ es primer, vale.
\item sino $n=a*b,\; 1>a, b<n+1$, por HI tanto $a$ como $b$ se factorizan.
\end{itemize}
Ahora verificamos la unicidad:\\
$$A=\{n\in{\mathbb{Z}}, n\text{ admite al menos 2 factores distintos}\}$$
Si $a\neq\emptyset\implies A$ tiene primer elemento
$$n = p_i^{e_1} \dots p_r^{e_r} = q_1^{t_q} \dots q_s^{t_s}$$
$$p_1|n = q_1^{t_1} \dots q_s^{t_s} \implies p_1|q_1 \text{ (alguno)}$$
\end{demo}

\begin{defi} si $n\in{\mathbb{Z}},\; n\neq 0 p$ es primo.
$$\mathcal{V}_p(n) = \max\{k\in{\mathbb{N}}\cup\{0\}\text{ tal que } p^k|n\}$$
Es el exponente de $p$ en la factorización de $n$.
\end{defi}

\begin{teo} Existen infinitos primos\end{teo}
\begin{demo}
Supongamos que: $p_1,\dots,p_N$ son todos los primos.
$$M = \prod^{N}p_i + 1 \in{\mathbb{N}}\implies\text{ se factoriza}\implies\text{ algún primo lo divide}$$
Luego $p_i | M + 1 \implies p_i|1$ {\bfseries absurdo}. 
\end{demo}

\begin{teo} si $a,b\in{\mathbb{Z}}$, no nulos:
$$ a|b \iff \mathcal{V}_p(a) \leq \mathcal{V}_p(b) \forall p \text{ primo.}$$
\end{teo}

\begin{demo}
\begin{itemize}
\item $\Rightarrow)$ si $a|b, \exists c \in{\mathbb{Z}}$ tal que $b=ac$ \\
Sea $p$ un primo 
\begin{align*}
\mathcal{V}_p(b) &= \mathcal{V}_p(ac) \\
&= \mathcal{V}_p(a) + \underbrace{\mathcal{V}_p(c)}_{\geq 0} \geq \mathcal{V}_p(a)
\end{align*}
\item $\Leftarrow)$Supongamos $a,b\in{\mathbb{Z}}$ tomo $c=\prod_{p}p^{\mathcal{V}_p(b)-\mathcal{V}_p(a)}$
$$ ac=(\prod_{p} p^{\mathcal{V}_p}) * (\prod_{p}p^{\mathcal{V}_p(b)-\mathcal{V}_p(a)}) = b$$
\end{itemize}
\end{demo}

\begin{teo} si $a,b\in{\mathbb{Z}}$ no nulos
$$\mcd(a,b)=\prod_{p}p^{\min\{\mathcal{V}_p(a), \mathcal{V}_p(b)\}}$$
\end{teo}

\begin{demo}
Por la proposición, si $d|a$\\

\[
\left. \begin{align*}
\mathcal{V}_p(d) &\leq \mathcal{V}_p(a) \\
\text{si } d|b,\; \mathcal{V}_p(d) &\leq \mathcal{V}_p(b)
\end{align*} \right\}
\text{ si } d|a \land d|b \to \mathcal{V}_p(d) \leq \min\{\mathcal{V}_p(a), \mathcal{V}_p(b)\}
\]

Claramente $\prod_{p}p^{\min\{\mathcal{V}_p(a), \mathcal{V}_p(b)\} |a \land |b \implies}$ es le mayor de los divisores comúnes.
\end{demo}

\part{Aritmética Modular}

\begin{defi} si $m\in{\mathbb{Z}},\; a,b\in{\mathbb{Z}}$, decimos que $a$ y $b$ son congruentes módulo $m$ (notación $a \equiv b\modu{m}$) si $m|a-b$\end{defi}

Como ejemplo si $m=0, a \equiv b \modu{0} \iff 0|a-b \iff a-b=0 \iff a=b$

\begin{teo} si $m\in{\mathbb{N}},\; a\in{\mathbb{Z}}\; r=$resto de dividir $a$ por $m$
$$ a \equiv r \modu{m} \quad a=mq+r$$
\end{teo}

\begin{demo}
$a\equiv r \modu{m} \iff m|a-r = mq$ \\
\begin{align*}
\intertext{Si $a,b\in{\mathbb{Z}, a\equiv b \modu{m} \iff}$}
r_m(a) &= r_m(b) \\
a &= mq_1+r_m(a) \\
b &= mq_2+r_m(b)
\end{align*}

\begin{itemize}
\item $\Leftarrow) a-b=b(q_1-q_2) \implies m|a-b$
\item $\Rightarrow) m|a-b = m(q_1-q_2)+r_m(a)-r_m(b) \implies m|r_m(a)-r_m(b)$
\begin{align*}
0 & \leq r_m(a) & < m &\\
-m & < r_m(a) - r_m(b) & < m &\\
0 & \geq -r_m(b) & > -m &
\end{align*}
$m|r_m(a)-r_m(b)$, si $r_m(a)-r_m(b)\neq{0}$ \\
$ m \leq r_m(a) - r_m(b)$ {\bfseries absurdo!} $\implies r_m(a) - r_m(b)=0 \implies r_m(a)=r_m(b)$ 
\end{itemize}
\end{demo}

\begin{obs}
Mirar congruencias modulo $m$ "es como" mirar el conjunto de restos en la división por $m$, con la ventaja que podemos operar.
\end{obs}

Fijemos $m\in{\mathbb{N}}$. En $\mathbb{Z}$ definimos una "relación" determinada por $a\sim b$ si $a\equiv b \modu{m}$

\begin{teo} La relación de congruencia es reflexiva, simetrica y transitiva\end{teo}

\begin{demo}
{\bfseries Reflexiva}: si $a\in{\mathbb{Z}}, a\sim a_m$ osea $a\equiv a\modu{m} \iff m|a-a = 0$ 
\end{demo}

\begin{demo}
{\bfseries Simetrica}: si $a\sim b\modu{m} \implies b\sim a\modu{m}$
$$ m|a-b \implies m|b-a = (a-b)(-1)$$
\end{demo}

\begin{demo}
{\bfseries Transitiva}: si $a\sim b\modu{m} \land b\sim c\modu{m} \implies a\sim c\modu{m}$
$$m|a-b \land m|b-c \implies m|a-c$$
$$m|(a-b)+(b-c) = a-c$$
\end{demo}

\begin{teo} si $a,b,c,d\in{\mathbb{Z}},\; a\equiv{b}\modu{m} \land c\equiv{d}\modu{m}$, entonces:\\
\begin{enumerate}
\item $a+c\equiv{b+d}\modu{m}$
\item $ac\equiv{bd}\modu{m}$
\end{enumerate}
\end{teo}

\begin{demo}
$$m|a-c \implies a-b = mq$$
$$m|c-d \implies c-d = mt$$

\begin{enumerate}
\item Quiero ver que:
\begin{align*}
m&|(a+c)-(b+d)\\
 &=(a-b)+(c-d)\\
 &=mq+mt = m(q+t)
\end{align*}
\item Quiero ver que:
\begin{align*}
m&|cd-bd\\
 &=cd-ad+ad-bd\\
 &=a(c-d)+d(a-b) = m(at+dq)
\end{align*}
\end{enumerate}
\end{demo}

Al tomar las congruencias en $m,\; m\neq{0}$, identificamos los números que tienen el mismo resto al dividirlos por $m$.

\begin{defi} Si dotas las clases$\neq\overline{0}$ tiene inverso multiplicativo, se dice que es {\itshape  cuerpo}.
$$\overline{x}\overline{y}=\overline{1}\quad \forall{\overline{x}\neq\overline{0}}$$
\end{defi}

\begin{teo} Si $a,b\in{\mathbb{Z}},\; a\equiv{b}\modu{m}$
$$\forall{n\in{\mathbb{N}}}\quad a^n\equiv{b^n}\modu{m}$$
\end{teo} TODO demostrar.

\section{Ecuaciones lineales de congruencia.}

Dados $m\in{\mathbb{Z}}\; m\neq{0}$ queremos estudiar la ecuación:
$$ax\equiv{b}\modu{m}\quad a,b\in{\mathbb{Z}}$$

Que $x$ sea solución quiere decir que:

$$a|ax-b \iff ax-b=my \iff ax-my=b$$

Vimos antes que $ax-my=b$ tiene solucón $\iff \mcd(a,m)|b$ (Ver teo. \ref{defdiafantica})

\begin{obs}
Notar que $x\equiv{x_0}\modu{\frac{m}{\mcd(a,m)}}$
\end{obs}

El numero $a\in{\mathbb{Z}}$ tiene inverso multiplicativo modulo $m$ si $\exists x\in{\mathbb{Z}}$ tal que $ax\equiv{1}\modu{m}$. Este tiene solucón $\iff\mcd(a,m)|1 \iff \mcd(a,m)=1$, osea $a$ y $m$ son coprimos.

\begin{teo} Si $m$ es primo $\forall i\neq{0}\in{\modu{m}}$ tiene inverso multiplicativo.\end{teo}

\begin{demo}
En general, $a$ tiene inverso módulo $m\iff\mcd(a,m)=1$.
$$ax\equiv1\modu{m}$$
si $m$ es primo, $a\in{\mathbb{Z}},\; \mcd(a,m) = \begin{cases} 1\text{ si }m\nmid a \\ m\text{ si } m|a\end{cases}$

\begin{itemize}
\item Si $\mcd(a,m)=1$, $a$ tiene inverso mód $m$
\item Si $\mcd(a,m)=m, a\equiv{0}\modu{m}$
\end{itemize}

En este caso, los números módulo $m$ son cuerpo.
\end{demo}

\begin{teo} si $p\in{\mathbb{N}}$ es primo, $a\in{\mathbb{Z}}$ tal que:
\begin{enumerate}
\item Si $p\nmid a,\; a^{p-1}\equiv1\modu{m}$
\item Siempre $a^p\equiv a\modu{m}$
\end{enumerate}
\end{teo}

\begin{demo}
En general $\{\overline{1},\overline{2},\overline{3},\dots, \overline{p-1},\}$ todos los no núlos módulo $p$ vimos que tienen inverso.
$$\underbrace{\{\overline{1},\overline{2},\overline{3},\dots, \overline{p-1},\}}_{(1)} \to^{\times{a}} \overbrace{\{a\times\overline{1},a\times\overline{2},a\times\overline{3},\dots, a\times\overline{p-1},\}}^{(2)}$$

Afirmo: Ambos conjuntos son iguales salvo el orden.\\
Primero veamos que (2) tiene $p-1$ elementos, osea:
$$ay\equiv aj\modu{p}\implies y\equiv j\modu{p}$$

Como $p\nmid a$, $a$ tiene inverso, si $ay\equiv aj\modu{p}, a^{-1}ay \equiv a^{-1}aj\modu{p} \implies y\equiv j\modu{p}$
Ademas los elementos de (2) no son divisibles por $p\implies(2)\subset(1)$ y ambos tienen $p-1$ elementos $\implies$ son iguales.

Luego como antes, si multiplicamos todos los elementos de (1) da lo mismo que hacerlo por (2)

$$ \overline{p-1}! \equiv a\times\overline{1}, a\times\overline{2}, \dots, a\times\overline{p-1} \equiv a^{-1}\times\overline{p-1}! \modu{p}$$

Al tener inverso multiplicativo, podemos cancelar $\implies a^{p-1}\equiv1\modu{p}$

Si $p|a,\quad a^p\equiv a \modu{p} \implies 0\equiv0\modu{p}$\\
Si $p\nmid a, \quad a^{p-1}\equiv1\modu{a} \implies a^p\equiv a\modu{p}$
\end{demo}

\begin{teo} (Euler - Fermant) si $m\in{\mathbb{N}}, d\in{\mathbb{Z}} \quad \mcd(a,m)=1$
$$a^{\phi(m)}\equiv1\modu{m}$$
Donde $\phi(m)=\#\{1\leq{i}\leq{m}, \mcd(i,m)=1\}$ es el Fi de euler.
\begin{itemize}
\item Si $p$ es primo, $\phi(p)=p-1$
\item Si $q,p$ primos distintos $\phi(qp)=qp-q-p+1$
\end{itemize}
\end{teo} TODO demostrar

\begin{teo} (Teorema Chino del resto) si $m,n\in{\mathbb{N}}$, el sistema de ecuaciones:

\begin{align}
\label{tcreja}x&\equiv a\modu{m}\\
\label{chrejb}x&\equiv b\modu{n}
\end{align}

Tiene solución única mñodulo $mn$ si $\mcd(m,n)=1$, en general hay solución única módulo $\mcd(m,n)$ si $\mcd(m,n)|a-b$
\end{teo} 

\begin{demo}
$x\equiv a\modu{m} \iff x=a+qm$, quiero ver cuales números satisfacen \ref{tcrejb}

$$a+qm\equiv b\modu{n} \iff \underbrace{qm\equiv b-a\modu{n}}_{\bigoplus}$$

Esta ecuación tiene solución $\iff \mcd(m,n)|b-a$ \\
Por euclides $\mcd(m,n)=mr+ns$ \\
Supongamos que son coprimos, $\mcd(m,n)=1\iff1\equiv mr\modu(n)$ \\
En $\bigoplus$ multiplico por $r$:
\begin{align*}
rmq &\equiv r(b-a)\modu{n} \\
q &\equiv r(b-a)\modu{n}
\end{align*}

\begin{align*}
q=r(b-a)+nt \implies x &= a+m[r(b-a)+nt]\\
                       &= a+mrb-mra+mnt\\
		       &= a(1-mr)+mrb+mnt\\
		       &= ans+bmr+mnt
\end{align*}

Osea $x\equiv ans + bmr \modu{mn}$

\begin{align*}
ns &\equiv 0\modu{n} & ns \equiv 1\modu{n}\\
mr &\equiv 1\modu{m} & mr \equiv 0\modu{m}
\end{align*}

$$ 1 = \frac{m}{\mcd(m,n)}r + \frac{r}{\mcd(m,n)}s \implies \frac{m}{\mcd(m,n)}r \equiv 1 \modu{\frac{n}{\mcd(m,n)}}$$

$$\mcd(m,n)|b-a \quad mq\equiv b-a\modu{n} \iff \frac{m}{\mcd(m,n)}q \equiv \frac{b-a}{\mcd(m,n)} \modu{\frac{n}{\mcd(m,n)}}$$

$$bq\equiv b-a\modu{n} \quad mq=(b-a)+nt \iff \frac{m}{\mcd(m,n)}q=\frac{b-a}{\mcd(m,n)} + \frac{n}{\mcd(m,n)}t$$

$$\iff \frac{mq}{\mcd(m,n)} \equiv \frac{b-a}{\mcd(m,n)} \modu{\frac{n}{\mcd(m,n)}}$$

Luego multiplicamos por $r$:

$$\underbrace{\frac{rm}{\mcd(m,n)}}_{=1}q \equiv \frac{r(b-a)}{\mcd(m,n)} \modu{\frac{n}{\mcd(m,n)}}$$

$$ x=a+mq = a+m(\frac{r(b-a)}{\mcd(m,n)} + \frac{n}{\mcd(m,n)}t)$$

$$q=\frac{r(b-a)}{\mcd(m,n)} + \frac{n}{\mcd(m,n)}$$

\end{demo}

\part{Números Complejos}

\begin{defi} El conjunto de los números complejos es el plano real $\mathbb{R}^2$ con las siguientes operaciónes:
\begin{itemize}
\item {\bfseries suma}: Como voctores, osea la soma es coordenada, coordenada
\item {\bfseries Producto}: El eje $x$ lo identificamos con $\mathbb{R}$ (con las operaciones de $\mathbb{R}$) en particular $(a,0)(a,0)=(ab,0)$ 
\end{itemize}
\end{defi}

\begin{defi} Notación: El vector $(a,b)$ lo notamos como $a+bi = a1+bi$\end{defi}

\begin{defi} Propiedad distributiva en $\mathbb{C}$
$$(a+bi)(c+di) = ac + adi + bci + bdi^2 = (ac-bd)+(ad+bc)i$$
\end{defi}

Con estas definiciones $(\mathbb{C}, +, *)$ es cuerpo.\\
La notación $Z=a+bi$ la llamamos {\itshape cartesiana} o {\itshape binomial}.\\
El valor "$a$" (la primera coordenada de $\mathbb{Z}$) se llama la parte {\bfseries real} de $Z$ (denotada $\re(Z)$), mientras que "$b$" es la parte imaginaria de $Z$ (denotada $\im(Z)$)

\begin{defi} Definimos la función conjugación:
\begin{align*}
\mathbb{C} &\to \mathbb{C}\\
Z &\to \overline{Z}\\
a+bi &\to a-bi\\
(a,b) &\to (a,-b)
\end{align*}
\end{defi}

\begin{defi} si $Z\in{\mathbb{C}}$, definimos su valor absoluto (lo que se mide) como
$$|Z| = \sqrt{a^2+b^2}$$
\end{defi}

\begin{teo} si $Z\in\mathbb{C}$
$$Z \overline{Z} = a^2+b^2$$
\end{teo}

\begin{demo}
\begin{align*}
Z\overline{Z} &= (a+bi)(a-bi)\\
              &= a^2-abi+abi-b^2i^2\\
	      &= a^2+b^2
\end{align*}
\end{demo}

Valen las siguientes propiedades:

\begin{teo} $\overline{z+w} = \overline{z} + \overline{w}$\end{teo}

\begin{teo} $\overline{zw} = \overline{z}\;\overline{w}$\end{teo}

\begin{teo} $z=\overline{z} \iff z\in{\mathbb{R}}$\end{teo}

\begin{teo} $z\overline{z}=|z|^2$\end{teo}

\begin{teo} $z+\overline{z} = 2\re(z)$\end{teo}

\begin{teo} $z-\overline{z} = 2\im(z)$\end{teo}

\begin{teo} $|zw| ) |z|\;|w|$\end{teo}

\begin{teo} $|z+w| \leq |z|+|w|$\end{teo} 

\begin{teo} $|z| = 0 \iff z=0$\end{teo}

% TODO: demostrar esto :V

\begin{obs}
La forma binomial de representar los números complejos es muy util para suar números complesjos, pero no para multiplicarlos.
\end{obs}

\begin{defi} {\bfseries Forma polar}:\\
Si $z=0\implies$ lo determinamos univocamente\\
Cualquier punto queda univocamente determinado por $\theta$ (la inclinación o ángulo) y su longitud.
\end{defi}

\begin{obs}
El ángulo lo mediamos en radianes con respecto a $x$, el mismo esta univocamente determinado, sino vale múltiplos de $2\pi$.
\end{obs}

En general para tener unicidad del ángulo, lo tomamos en $[0, 2\pi)$.

Si $z\in{\mathbb{C}}, z\neq{0}$

$$ z= \underbrace{|z|}_{r} \underbrace{\frac{z}{|z|}}_\text{Tiene valor absoluto 1} \quad \left|\frac{z}{|z|}\right| = \frac{|z|}{||z||} = 1$$

\[
\left.
\begin{align*}
\sin(\theta) &= \im\left(\frac{z}{|z|}\right)\\
\cos(\theta) &= \re\left(\frac{z}{|z|}\right)
\end{align*}
\right\}
(r,\theta) \to r(\cos(\theta) + i\sin(\theta))
\]

Al revés, dado $Z=a+bi$, ¿como calcular $(r,\theta)$? \\
$r=|Z|$, para calcular $\theta,\; \cos(\theta)=\frac{z}{|z|}$ \\
Esta condición determina $\theta$ salvo el signo, si $z$ está en el $1^\text{er}$ o $2^\text{do}$ cuadrante, $\theta=\arccos\left(\frac{z}{|z|}\right)$, sino $\theta=-\arccos\left(\frac{z}{|z|}\right)$

\begin{obs} Es posible escribir un numero complejo usando notación de Eueler:
$$ z=\underbrace{|z|}_{r}\underbrace{(\cos(\theta)+i\sin(\theta))}_{e^{i\theta}}$$
\end{obs}



\part{Grafos}

\end{document}
