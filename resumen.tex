\documentclass[9pt,a4paper]{article}


\usepackage[utf8]{inputenc}
\usepackage{amsthm}
\usepackage{mathtools}
\usepackage{amsfonts}

\title{Resumen matemática discreta 1}
\author{Andrés Villagra de la Fuente - {\bfseries  andres.villagra(at)alumnos.unc.edu.ar}}

\theoremstyle{definition}
\newtheorem{defi}{Definición}
\newtheorem{axi}{Axioma}
\newtheorem{lema}{Lema}

\theoremstyle{plain}
\newtheorem{teo}{Teorema}
\newtheorem{demo}{Demostración}[teo]
\newtheorem{col}{Colonario}
\newtheorem{obs}{Observación}
\begin{document}

\maketitle

\part{Números Enteros}

\section{Introducción}

\begin{defi} Los números naturales son un conjunto con una {\bfseries función  sucesor} que satisface los siguientes axiomas: \end{defi}

\begin{axi} \label{natual1}1 es un número Natural. \end{axi}
\begin{axi} \label{suc}Todo número tiene un sucesor\end{axi}
\begin{axi} \label{sucind}La función sucesor es injectiva ($\forall x \forall y, suc(x) \neq suc(y)$ si $x \neq y$)\end{axi}
\begin{axi} \label{primer1}El 1 no es sucesor de nadie\end{axi}
\begin{axi} \label{cnmp}si $S$ es un conjunt con una función sucesor, $1 \in S$ y satisface los axiomas \ref{natural1}-\ref{primer1} entonces es un conjunto natural\end{axi}

Por el axioma \ref{cnmp} podemos ver que el conjunto de los naturales es el conjunto de números mas pequeño con estás propiedades.

\begin{obs}
\begin{equation*}
\underbrace{|\mathbb{N}\cup\{\dots;\frac{-3}{2};\frac{-2}{2};\frac{-1}{2};\frac{1}{2};\frac{2}{2};\frac{3}{2};\dots\}}_\text{Satisface A\ref{natural1}-A\ref{primer1} pero no A\ref{cnmp}}|>|\mathbb{N}|
\end{equation*}
\end{obs}

\section{Aritmética}

En esta sección daremos definición a las diferentes operaciónes aritméticas y sus diferentes propiedades.

Para definir la suma damos uso del principio de inducción (definición \ref{defindux}) para definirla en $\mathbb{N}$

\begin{defi} \label{defsuma} 
\begin{align*}
\text{suma}(m,n) &\doteq \text{ si } n=1, \text{suc}(m) \\
                 &\doteq \text{ si no } n = \text{suc}(n-1) \implies \text{suma}(\text{suc}(m),n-1)                
\end{align*}
\end{defi}

\begin{defi}\label{defmq} si $m,n\in\mathbb{N}$, decimos que $n<m$ si $\exists J\in\mathbb{N}$ tal que $m=n+J$\end{defi}

\begin{teo} si $m,n\in\mathbb{N}$, entonces $n>m \lor n<m \implies m\neq n$\end{teo}

\begin{demo}

\begin{align*}
\text{Si }& m<n \land m=n; m,n \in{\mathbb{N}} \\
          & \exists L \text{ tal que } n=m+L   \\
          & \text{Ya que } m=n \implies L=0    \\
	  & L\notin\mathbb{N}                  \\           
	  & \text{Absurdo.}
\end{align*}

\end{demo}

Para definir el producto damos uso tambien de una definición recursiva.

\begin{defi} \label{defprod}
\begin{align*}
\text{prod}(m,n) &\doteq \text{si } n=1    \implies \text{prod}(m,1)=m \\
                 &\doteq \text{si } n\neq1 \implies \text{prod}(m,n)=\text{prod}(m,n-1)+m
\end{align*}
\end{defi}

\begin{obs}
Al definir así podesmo ver tambien que:\\
prod$(m,n)=$ prod$(m,n-1)+m$\\
Por ende la propiedad distributiva
\begin{align*}
prod(a,suma(b,c))=&suma(prod(a,b),prod(a,c))\\
&a+prod(a,suma(b,c)-1)\\
&\text{por siguiente}\\
&prod(a,b)+prod(a,sum(b,c)-b)\\
&prod(a,b)+prod(a,c)
\end{align*}
\end{obs}

\begin{lema}
Si $m,n,l\in\mathbb{N}$, tal que $l<m\land m<n\implies l<n$ (transitiva)

\begin{align*}
& m = a+l   & a\in\mathbb{N} \\
& n = b+m   & b\in\mathbb{N} \\
& n = \underbrace{b+a}_{\in{\mathbb{N}}}+l \implies n<l
\end{align*}
\end{lema}

\section{inducción}

\begin{axi}\label{defindN} Principio de inducción\\
Suponiendo que $\forall x\in{\mathbb{N}}$ tenemos una afirmación $P(x)$, y queremos provar su veracidad.\\
Si valen:\\
$1^{er}$ caso: $P(1)$ es verdadera
Paso Inductivo: si $P(x)$ es verdadera $\implies P(x+1)$ tambien debe ser verdadera $\forall x\in{\mathbb{N}}$
\end{axi}

El principio de inducción nos permite demostrar que una proposición es valida en todo el conjunto de $\mathbb{N}$, además podemos comprobar una proposición en $\mathbb{Z}$ (ver teo. \ref{defindZ})

\begin{defi} Sucesiones recursivas:\\
Dado $(a_n), n\in\mathbb{N}$ una sucesión de números.\\
$$S_k = \sum_{i=1}^{k}a_i = a_1 + a_2 + \dots + a_k \forall k\in{\mathbb{N}}$$
\end{defi}

Un ejemplo de definición de sucesión recursiva

\begin{align*}
S_1 = & a_1 \\
S_k = & S_{k-1} + a_k & k>1
\end{align*}

Afirmo que está bien definida:\\

$$S=\{k\in{\mathbb{N}}: S_k \text{ esta bien definida}\}$$

Quiero probar que $S=\mathbb{N}$

\begin{align*}
& 1 \in S \\
& \text{Si } k\in{S}\implies k+1\in{S}\\
& \underbrace{\mathbb{N}\subseteq S \subseteq\mathbb{N}}_{\text{Por el A\ref{cnmp}}\implies S=\mathbb{N}}
\end{align*}

\begin{defi} Inducción corrida \\ {\bfseries TODO}  \end{defi}

\begin{teo} \label{teobuenorden}Todo subconjunto no vacio de $\mathbb{N}$ tiene primer (o mínimo) elemento \end{teo}

\begin{demo}Bucar una buena demo\end{demo}

\section{Propiedades de $\mathbb{N}$}

$(\mathbb{N}, +, \bullet)$ es un {\itshape trinoide} y satisface las siguientes propiedades:

$(\mathbb{N}, +)$ vale:

\begin{teo} Es asociativo:\\
\begin{align*} & (a+b)+c=a+(b+c) & \forall a,b,c\in{\mathbb{N}} \end{align*}
\end{teo}

\begin{teo} Es conmutativa:\\
\begin{align*} & m+n=n+m & \forall m,n\in{\mathbb{N}} \end{align*}
\end{teo}

$(\mathbb{N}, \bullet)$ vale:

\begin{teo} Es asociativa:\\
\begin{align*} & l*(m*n)=(l*m)*n & \forall l,m,n\in{\mathbb{N}} \end{align*}
\end{teo}

\begin{teo} Es conmutativo:\\
\begin{align*} & l*m=m*l & \forall l,m\in{\mathbb{N}} \end{align*}
\end{teo}

\begin{teo} Existe el elemento neutro:\\
\begin{align*} & 1*l=l & \forall l\in{\mathbb{N}} \end{align*}
\end{teo}

TODO: demostrar estos teoremas.
\part{Conteo}

\part{Divisibilidad}

\part{Aritmética Modular}

\part{Números Complejos}

\part{Grafos}

\end{document}
