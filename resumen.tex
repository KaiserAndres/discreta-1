\documentclass[9pt,a4paper]{article}


\usepackage[utf8]{inputenc}
\usepackage{amsthm}
\usepackage{mathtools}
\usepackage{amsfonts}

\title{Resumen matemática discreta 1}
\author{Andrés Villagra de la Fuente - {\bfseries  andres.villagra(at)alumnos.unc.edu.ar}}

\theoremstyle{definition}
\newtheorem{defi}{Definicion}
\newtheorem{axi}{Axioma}

\theoremstyle{plain}
\newtheorem{teo}{Teorema}
\newtheorem{col}{Colonario}
\newtheorem{obs}{Observación}
\begin{document}

\maketitle

\part{Números Enteros}

\begin{defi} Los números naturales son un conjunto con una {\bfseries función  sucesor} que satisface los siguientes axiomas: \end{defi}

\begin{axi} \label{natual1}1 es un número Natural. \end{axi}
\begin{axi} Todo número tiene un sucesor\end{axi}
\begin{axi} La función sucesor es injectiva ($\forall x \forall y, suc(x) \neq suc(y)$ si $x \neq y$)\end{axi}
\begin{axi} \label{primer1}El 1 no es sucesor de nadie\end{axi}
\begin{axi} \label{cnmp}si $S$ es un conjunt con una función sucesor, $1 \in S$ y satisface los axiomas \ref{natural1}-\ref{primer1} entonces es un conjunto natural\end{axi}

Por el axioma \ref{cnmp} podemos ver que el conjunto de los naturales es el conjunto de números mas pequeño con estás propiedades.

\begin{obs}
\begin{equation*}
\underbrace{|\mathbb{N}\cup\{\dots;\frac{-3}{2};\frac{-2}{2};\frac{-1}{2};\frac{1}{2};\frac{2}{2};\frac{3}{2};\dots\}}_\text{Satisface A\ref{natural1}-A\ref{primer1} pero no A\ref{cnmp}}|>|\mathbb{N}|
\end{equation*}
\end{obs}

\part{Conteo}


\part{Divisibilidad}

\part{Aritmética Modular}

\part{Números Complejos}

\part{Grafos}

\end{document}
